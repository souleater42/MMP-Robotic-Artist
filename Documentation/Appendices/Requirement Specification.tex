\documentclass{article}

\usepackage{amsmath}
\usepackage{gensymb}
\usepackage{parskip}
\usepackage[none]{hyphenat}

\usepackage[bottom]{footmisc}

\usepackage[margin=1in, left=1.5in, includefoot]{geometry} 


%graphic stuff

\usepackage{graphicx} % allows images

\usepackage{float} %helps with posisioning


%header and footer stuff

\usepackage{fancyhdr}

\pagestyle{fancy}

\fancyhead{}

\fancyfoot{}

\fancyhead[L]{Robotic Artist:Walter / 0.1 (Final)}

\fancyfoot[L]{Aberystwyth University / Computer Science}

\fancyfoot[R]{page: \thepage}

\renewcommand{\headrulewidth}{0pt}

\renewcommand{\footrulewidth}{0pt}


%begins document

 \begin{document}



    % begin title screen

    \begin{titlepage}

        \begin{center}

        \line(1,0){340}\\ 


        \large{\bfseries Robotic Artist:Walter} \\

        \large {\bfseries G400 Computer Science, CS39440 }\\
        
        \large {\bfseries Requirements Specification}\\


         \line(1,0){250}\\

         \textsf {Author: Matthew Howard \\
          Candidate Number: 150035512\\
          User ID: Mah60 \\
          Supervisor Name: Helen Miles \\
          Supervisor ID: Hem23\\
          Last Modified: 12/02/2018 \\
          Version: 1.0\\
          Status: Final} \\

        \end{center}        

    \end{titlepage}
  
    \clearpage

 	\tableofcontents
 	
 	\clearpage

    \section{Introduction}

This document will talk about the requirements that are set for the final product of the Robotic Artist is going to have. The requirements will be expanded on to clearly explain what is wanted from this project. Each of the requirements will also be assigned a priority and explain the resoning behind it.

	\section{Requirements}
	
The following list is the requirements that are set for this project. If these tasks are completed, then the Robotic Artist should work as wanted.

\begin{enumerate}
  \item Connect and control Roland 990 pen plotter
  \item Control a camera
  \item Create a Graphical Unit Interface(GUI)
  \item Program Raspberry Pi to become a Router between devices
  \item Generate artistic styles
  \item Plot artistic drawings
  \item Make robot easy to use
  \item Admins has control on run time
\end{enumerate}

The following section will expand on each requirement and talk about the priority of having these requirements working.\\ \newline

\textbf{Connect and control Roland 990 pen plotter} \\
\textbf{Control a camera} \\
\textbf{Create a Graphical Unit Interface(GUI)}\\
\textbf{Program Raspberry Pi to become a Router between devices}\\
\textbf{Generate artistic styles}\\
\textbf{Plot artistic drawings}\\
\textbf{Make robot easy to use}\\
\textbf{Admins has control on run time}\\



    \section{Versions}


\begin{center}

\begin{tabular}{| l | p{8cm} | p{3cm}|}

\hline

\textbf{Version} & \textbf{Description} & \textbf{Date Modified} \\\hline

0.1 & blank. & time, date \\ \hline


\end{tabular}

\end{center}

 \end{document}