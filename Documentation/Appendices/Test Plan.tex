\documentclass{article}

\usepackage{lscape}

\usepackage[margin=1in, left=1.5in, includefoot]{geometry} 


%graphic stuff

\usepackage{graphicx} % allows images

\usepackage{float} %helps with posisioning

\usepackage{longtable}


%header and footer stuff

\usepackage{fancyhdr}

\pagestyle{fancy}

\fancyhead{}

\fancyfoot{}

\fancyhead[L]{Robotic Artist:Walter / 0.3 (WIP)}

\fancyfoot[L]{Aberystwyth University / Computer Science}

\fancyfoot[R]{page: \thepage}

\renewcommand{\headrulewidth}{0pt}

\renewcommand{\footrulewidth}{0pt}
\usepackage{amsmath}
\usepackage{gensymb}
\usepackage{parskip}
\usepackage[none]{hyphenat}
%begins document

 \begin{document}



    % begin title screen

    \begin{titlepage}

        \begin{center}

        \line(1,0){340}\\ 


        \large{\bfseries Robotic Artist:Walter} \\

        \large {\bfseries G400 Computer Science, CS39440 }\\
        
        \large {\bfseries Test Plan}\\


         \line(1,0){250}\\

         \textsf {Author: Matthew Howard \\
          Candidate Number: 150035512\\
          User ID: Mah60 \\
          Supervisor Name: Helen Miles \\
          Supervisor ID: Hem23\\
          Last Modified: 26/04/2018 \\
          Version: 0.3\\
          Status: Work In Progress(WIP)} \\

        \end{center}        

    \end{titlepage}
  
    \clearpage

     \tableofcontents
     
     \clearpage

    \section{Introduction}
This document is going to talk about how to test the Robotic Artist product. These tests include unit tests and the testing specifications that will be used to test the product. This test plan will need to be kept up to date with future development and must make sure that all requirements from the Requirement Specifications are followed.\cite{Requirement_specifications}\\ \newline
    \section{Unit tests}
Unit tests allow you to test source code that you would not usually be able to test through visual methods. They work by creating a simulated environment so that you can then test the individual methods and see if they give the expected outcomes.\\ \newline
The unit tests are programmed in Python 2.7 using the library Unittest\cite{Unittest}. To run the unit tests all together, within the terminal run the command 'python run\_tests.py'. Within this file is a Test Suite that will run all the unit tests developed for this project. For future development make sure to create unit tests for code that requires it and add the class that is made to the test suite within run\_tests.py.
\\ \newline
    \section{Testing specifications}
This section is about testing the code that cannot be tested using unit testing and making sure that all the requirements have been met. These are the Events that occur in the GUI, the plotter plotting what is shown in the GUI and that the ImageProcessor is giving out the correct style of what is wanted.\\ \newline
There are going to be three tables to the testing specification. When testing the product the results will be recorded and written up to be reviewed. The tables can be seen below.\\ \newline
\clearpage
\begin{landscape}
\newcounter{id}
\stepcounter{id} % set counter to one
\begin{center}
\subsection{GUI tests}
\begin{longtable}{| l | p{3cm} | p{4cm}| p{4cm}| p{4cm} | l | l |}

\hline

\textbf{ID} & \textbf{Test} & \textbf{Test Instructions} & \textbf{Expected Outcome} & \textbf{Actual Outcome} & \textbf{Date} & \textbf{Tester ID}\\\hline

\arabic{id} \stepcounter{id} & Open Application & Open the application. Do this 5 times. & The application will open. & & &  \\ \hline

\arabic{id} \stepcounter{id} &  Exit application function & Go to file menu and click 'Exit Application'. Try 3 times. Then try again with the command 'Ctrl+Q'. & The function will open a pop-up, asking if you are sure. Then when yes is pressed the application will close and if no is pressed you will be returned to the page you were on. & & &  \\ \hline

\arabic{id} \stepcounter{id} & Restart application & Test the function restart application 2 times on each page that is inside the file menu.& The application will be restart to the video capture page. If on the video capture page, a pop-up will tell you saying that you cannot restart on this page.& & &  \\ \hline

\arabic{id} \stepcounter{id} & Admin login function & go to Admin and then click 'login', do 2 times and then try to log in. & A pop-up window will appear that allows the user to log in.& & &  \\ \hline

\arabic{id} \stepcounter{id} & Toggle admin view function & Go to the Admin menu and click 'toggle admin view'. Do this 3 times & This should open a window that allows the admin to view the input and output of the program. What is occurring on the application at runtime? & & &  \\ \hline

\arabic{id} \stepcounter{id} & Tutorial function & Go to Help menu and click 'tutorial'. Do this 3 times. & The application will open a window that displays instructions on how to use the application given.& & &  \\ \hline

\arabic{id} \stepcounter{id} & Video Capture & Open the application 3 times and check if the video display is working correctly. & The video capture will be displayed with no error. & & &  \\ \hline

\arabic{id} \stepcounter{id} & Checkbox - allow to use photo in the application & Click the 'Capture Picture' button. Do this 2 times. Then check the checkbox and see if you are allowed to proceed. Do this 2 times. & when unticked an error message will pop up. When checked user will go to the picture acceptance page. & & &  \\ \hline

\arabic{id} \stepcounter{id} & Capture Picture button & Check checkbox 1 and Click the 'Capture Picture' button 4 times. & The user will be taken to the picture acceptance page. The video capture thread has stopped and the picture that was taken is displayed on the picture acceptance page.& & &  \\ \hline

\arabic{id} \stepcounter{id} & Click No; picture acceptance page & Click the no button on the picture acceptance page. Do this 2 times. & When 'no' is pressed a pop-up will turn up. Which ask's if the user definitely wants to reject the taken picture. & & &  \\ \hline

\arabic{id} \stepcounter{id} & Click Yes; picture acceptance page & Click the yes button on the picture acceptance page. Do this 2 times. & This should take you to the style selection page. & & &  \\ \hline

\arabic{id} \stepcounter{id} & Select no styles & Click the continue button without selecting any styles.Do this 2 times. & A error message should pop-up, saying no styles have been selected. & & &  \\ \hline

\arabic{id} \stepcounter{id} & Select Multiple styles & Click the continue button with selecting multiple styles.Do this 2 times. & A error message should pop-up, saying too many styles have been selected & & &  \\ \hline

\arabic{id} \stepcounter{id} & Select a single style & Click the continue button with selecting only one stlye.Do this for 2 times. & You should be taken to the style acception page. With your picture processed in the style you selected. & & &  \\ \hline

\arabic{id} \stepcounter{id} & Click No; style confirmation page & Click the no button on the style confirmation page. Do this 2 times. & When 'no' is pressed a pop-up will turn up. Which asks if your sure that you done want this picture. & & &  \\ \hline

\arabic{id} \stepcounter{id} & Click Yes; style confirmation page & Click the yes button on the style confirmation page. Do this 2 times. & This should start the plot for the style that was confirmed to be correct. & & &  \\ \hline


\end{longtable}
\end{center}
 \clearpage
 \begin{center}
\subsection{Image Processor tests}

\begin{longtable}{| l | p{3cm} | p{4cm}| p{4cm}| p{4cm} | l | l |}

\hline

\textbf{ID} & \textbf{Test} & \textbf{Test Instructions} & \textbf{Expected Outcome} & \textbf{Actual Outcome} & \textbf{Date} & \textbf{Tester ID}\\\hline

\arabic{id} \stepcounter{id} & Dithering style & select the dithering style and click the continue button. Do this 3 times. & The dithering style will be applied to the picture that was taken from the video capture. & & &  \\ \hline

\arabic{id} \stepcounter{id} & Edge style & select the Edge style and click the continue button. Do this 3 times. & The Edge style will be applied to the picture that was taken from the video capture. & & &  \\ \hline

\end{longtable}
\end{center}
 \clearpage
 \begin{center}
\subsection{Plotter tests}

\begin{longtable}{| l | p{3cm} | p{4cm}| p{4cm}| p{4cm} | l | l |}

\hline

\textbf{ID} & \textbf{Test} & \textbf{Test Instructions} & \textbf{Expected Outcome} & \textbf{Actual Outcome} & \textbf{Date} & \textbf{Tester ID}\\\hline

\arabic{id} \stepcounter{id} & Dithering plotter & Once Image has been processed. Press 'yes' on the style confirmation page. Do 2 times. & The plotter should go through each coordinate and put the pen up and down at the respective locations. When the coordinate has a point the right of it, then pen moves to it instead of pulling the pen up and down again. & & &  \\ \hline

\arabic{id} \stepcounter{id} & Edge plotter & Once Image has been processed. Press 'yes' on the style confirmation page. Do 2 times. & The plotter should go to a point and draw a line connecting all the points that are next to that point. & & &  \\ \hline

\end{longtable}
\end{center}

 \clearpage
\end{landscape}

    \section{Versions}

\begin{center}

\begin{tabular}{| l | p{8cm} | p{3cm}|}

\hline

\textbf{Version} & \textbf{Description} & \textbf{Date Modified} \\\hline

0.1 & Created basic version of the text for unit test and the testing specifications. Then created blank tables. Need to fill in. & 25/04/18 \\ \hline
0.2 & Created testing specifications tables, these go through each test that is needed to be done on the GUI, image processors and the plotter. & 26/04/18\\ \hline
0.3 & Corrected spelling and grammar errors. Changed style acceptance page to style confirmation page. & 27/04/18\\ \hline


\end{tabular}

    \begin{thebibliography}{9}

    \bibitem{Requirement_specifications}

    Matthew Howard (mah60);Requirement Specifications; 25/04/2018 \\ 
    
    This document talks about the requirements that have been given to robotic artist project. The document can be found in the Appendices for this project.\\ 
    
    \bibitem{Unittest}
    
    Python Software Foundation; Unittest; 26/04/2018\\
    
    \textit{https://docs.python.org/3/library/unittest.html}
    
    This is the documentation for the Unittest library for python. This was the library that is used for doing unit tests in this project.
    \end{thebibliography}

\end{center}

 \end{document}