\documentclass{article}


\usepackage[margin=1in, left=1.5in, includefoot]{geometry} 


%graphic stuff

\usepackage{graphicx} % allows images

\usepackage{float} %helps with posisioning


%header and footer stuff

\usepackage{fancyhdr}

\pagestyle{fancy}

\fancyhead{}

\fancyfoot{}

\fancyhead[L]{Robotic Artist:Walter / 0.1 (Final)}

\fancyfoot[L]{Aberystwyth University / Computer Science}

\fancyfoot[R]{page: \thepage}

\renewcommand{\headrulewidth}{0pt}

\renewcommand{\footrulewidth}{0pt}


%begins document

 \begin{document}



    % begin title screen

    \begin{titlepage}

        \begin{center}

        \line(1,0){340}\\ 


        \large{\bfseries Robotic Artist:Walter} \\

        \large {\bfseries G400 Computer Science, CS39440 }\\
        
        \large {\bfseries Final Report}\\


         \line(1,0){250}\\

         \textsf {Author: Matthew Howard \\
          Candidate Number: 150035512\\
          User ID: Mah60 \\
          Supervisor Name: Helen Miles \\
          Supervisor ID: Hem23\\
          Last Modified: 12/02/2018 \\
          Version: 1.0\\
          Status: Final} \\

        \end{center}        

    \end{titlepage}
  
    \clearpage

 	\tableofcontents
 	
 	\clearpage

    \section{Introduction}


    \begin{thebibliography}{9}

    \bibitem{Paul_the_robot}

    EXAMPLE: Patrick Tresset, Frederic Fol Leymarie, Goldsmiths College ; "Portrait drawing by Paul the robot"; 01/02/2018 \\ 

    \textit{http://doc.gold.ac.uk/~ma701pt/patricktresset/wp-content/uploads/2015/03/computerandgraphicstresset.pdf}

    

This is the main article that I will be using throughout the project, as it was where the idea from the project came from and talks about different methods and styles when it comes to developing robotic artists. Such as, splitting the image into different sections and developing different consistences to pick out the lines and the shadows. Also it has references to other projects that are just like this one. 
    

    \end{thebibliography}

    \section{Versions}


\begin{center}

\begin{tabular}{| l | p{8cm} | p{3cm}|}

\hline

\textbf{Version} & \textbf{Description} & \textbf{Date Modified} \\\hline

0.1 & blank. & time, date \\ \hline


\end{tabular}

\end{center}

 \end{document}